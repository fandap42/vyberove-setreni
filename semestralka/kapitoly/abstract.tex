\begin{abstract}
Tato práce se zabývá srovnáním metod pro odhad rozptylu výběrových kvantilů, což je klíčový problém v mnoha oblastech statistiky, zejména při práci s daty, která neodpovídají normálnímu rozdělení. Hlavním cílem je porovnat přesnost a spolehlivost asymptotického přístupu založeného na Taylorově rozvoji (využívajícího jádrové odhady hustoty) s neparametrickou metodou Bootstrap.
Za účelem srovnání byla provedena rozsáhlá Monte Carlo simulační studie na datech z log-normálního rozdělení s různou mírou asymetrie ($\sigma \in \{0.5, 1.0, 1.5\}$) a pro různé rozsahy výběru ($n \in \{30, 100, 250, 500, 1000, 1500, 2000\}$). Studie se zaměřila na odhad rozptylu jak pro medián, tak pro extrémní kvantily ($p=0.99$).
Výsledky ukazují, že zatímco pro symetrická rozdělení a centrální kvantily poskytuje Plug-in metoda (Taylorův rozvoj) uspokojivé výsledky, v případě silně sešikmených dat a extrémních kvantilů dramaticky selhává a podhodnocuje skutečnou variabilitu. Naproti tomu metoda Bootstrap vykazuje výrazně vyšší robustnost a přesnost pokrytí intervalů spolehlivosti, ačkoliv je výpočetně náročnější. Práce proto doporučuje použití Bootstrapu pro inferenci o extrémních kvantilech v nesymetrických rozděleních.
\end{abstract}
