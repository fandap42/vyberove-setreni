\chapter{Úvod}
Odhadování kvantilů a jejich přesnosti je klíčovou úlohou v mnoha oblastech statistiky.. Zatímco bodový odhad kvantilu pomocí výběrového kvantilu je relativně přímočarý, odhad jeho rozptylu (a tím i konstrukce intervalů spolehlivosti) představuje náročnější problém, zejména pokud neznáme rozdělení, ze kterého data pocházejí, nebo pokud je toto rozdělení výrazně sešikmené.

Cílem této práce je porovnat různé metody odhadu rozptylu výběrových kvantilů. Zaměříme se na tři přístupy: teoretický asymptotický rozptyl založený na Taylorově rozvoji (který vyžaduje znalost hustoty), praktickou  \uv{plug-in} metodu využívající jádrový odhad hustoty, a neparametrický bootstrap.

Jako modelové rozdělení pro naši simulační studii jsme zvolili log-normální rozdělení. Toto rozdělení je v praxi velmi časté (např. v příjmovém rozdělení) a vyznačuje se silnou asymetrií a těžkými chvosty, což může činit problémy asymptotickým aproximacím, zejména při malém rozsahu výběru nebo při odhadu extrémních kvantilů.

V následující kapitole nejprve teoreticky odvodíme asymptotický rozptyl výběrového kvantilu. Následně popíšeme design simulační studie, prezentujeme výsledky pro různé rozsahy výběrů a hladiny kvantilů a v závěru diskutujeme vhodnost jednotlivých metod.
