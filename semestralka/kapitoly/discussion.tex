\chapter{Diskuze}
Výsledky simulace potvrzují, že odhad rozptylu extrémních kvantilů ($p=0.99$) pomocí asymptotických vzorců je v případě sešikmených dat rizikový.

\section{Analýza selhání Plug-in metody}
Zatímco pro kvantil $p=0.95$ a 95\% interval spolehlivosti metoda funguje uspokojivě, při požadavku na vysokou spolehlivost (99\% interval pro kvantil $p=0.99$) a vysoké asymetrii ($\sigma=1.5$) dochází k selhání. Odhadnutá hustota $\hat{f}$ je systematicky vychýlená, což vede k podhodnocení směrodatné chyby až o desítky procent, jak ukazuje graf relativního vychýlení. Výsledné intervaly jsou příliš úzké, a proto nepokrývají skutečnou hodnotu s požadovanou pravděpodobností 0.99.

\section{Doporučení}
Pro praxi doporučujeme:
\begin{itemize}
    \item Pro běžné kvantily ($p \le 0.95$) lze využít Plug-in metodu, pokud je $n$ dostatečně velké ($n > 100$).
    \item Pro extrémní kvantily ($p=0.99$) a konstrukci intervalů s vysokou spolehlivostí (99 \%) je nutné použít robustnější metody, jako je Bootstrap, nebo metody založené na Teorii extrémních hodnot (EVT), protože standardní asymptotická aproximace v těchto oblastech selhává.
\end{itemize}
