\chapter{Diskuze}
Výsledky simulace potvrzují, že odhad rozptylu extrémních kvantilů ($p=0.99$) pomocí asymptotických vzorců je v případě sešikmených dat rizikový.

\section{Analýza selhání Plug-in metody}
Klíčovým zjištěním práce je selhání Plug-in metody (založené na jádrovém odhadu hustoty) v oblastech s nízkou pravděpodobností výskytu. Tento jev lze vysvětlit pomocí tzv. \textbf{Bias-Variance tradeoff} při volbě vyhlazovacího okna (bandwidth) u KDE.
Standardní metody pro volbu šířky okna (např. Scottovo pravidlo, které jsme použili) jsou optimalizovány pro minimalizaci globální chyby (IMSE) přes celý nosič rozdělení. U asymetrických rozdělení, jako je log-normální, je tato globální volba kompromisem, který vede k "přehlazení" (oversmoothing) v oblasti chvostů.
Důsledek pro odhad rozptylu kvantilu je fatální:
\begin{enumerate}
    \item KDE v chvostu nadhodnocuje hustotu ($E[\hat{f}(x)] > f(x)$), protože "rozmazává" pravděpodobnostní hmotu z centra do chvostů.
    \item Protože rozptyl kvantilu je nepřímo úměrný čtverci hustoty ($AVar \propto 1/f^2$), nadhodnocení hustoty vede k výraznému \textbf{podhodnocení rozptylu}.
    \item Výsledkem jsou nerealisticky úzké intervaly spolehlivosti, které nedokáží pokrýt skutečnou hodnotu s požadovanou pravděpodobností.
\end{enumerate}
Tento metodický nedostatek je u těžkých chvostů principiální a nelze jej snadno odstranit pouhým zvětšením rozsahu výběru, jak ukazují naše výsledky pro $n=1000$.

\section{Doporučení}
Pro praxi doporučujeme:
\begin{itemize}
    \item Pro běžné kvantily ($p \le 0.95$) lze využít Plug-in metodu, pokud je $n$ dostatečně velké ($n > 100$).
    \item Pro extrémní kvantily ($p=0.99$) a konstrukci intervalů s vysokou spolehlivostí (99 \%) je nutné použít robustnější metody, jako je Bootstrap, nebo metody založené na Teorii extrémních hodnot (EVT), protože standardní asymptotická aproximace v těchto oblastech selhává.
\end{itemize}
