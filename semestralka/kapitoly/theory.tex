\chapter{Teoretická část}
V této kapitole se zaměříme na odvození asymptotického rozptylu výběrového kvantilu. Toto odvození je bude dále použito jako základ metod pro odhad rozptylu kvantilů ve výběrových šetřeních.

\section{Definice a značení}
Nechť $X_1, X_2, \dots, X_n$ je náhodný výběr z rozdělení se spojitou distribuční funkcí $F(x)$ a hustotou pravděpodobnosti $f(x)$.
Definujme $p$-tý teoretický kvantil $q_p$ jako hodnotu, pro kterou platí:
\begin{equation}
    F(q_p) = p, \quad \text{kde } p \in (0, 1).
\end{equation}
Výběrový kvantil $\hat{q}_p$ je definován pomocí empirické distribuční funkce $\hat{F}_n(x)$ jako \parencite{Hyndman1996}:
\begin{equation}
    \hat{q}_p = \hat{F}_n^{-1}(p) = \inf \{x: \hat{F}_n(x) \ge p\}.
\end{equation}

Pro účely této práce budeme uvažovat Log-normální rozdělení $LN(\mu, \sigma^2)$, jehož hustota je dána:
\begin{equation}
    f(x) = \frac{1}{x\sigma\sqrt{2\pi}} e^{-\frac{(\ln x - \mu)^2}{2\sigma^2}}, \quad x > 0.
\end{equation}

\section{Odvození asymptotického rozptylu (Delta metoda)}
Pro formální odvození asymptotického rozdělení výběrového kvantilu využijeme tzv. Delta metodu aplikovanou na kvantilovou funkci.
Označme $F$ distribuční funkci a $Q(p) = F^{-1}(p)$ kvantilovou funkci.
Výběrový kvantil $\hat{q}_p$ lze chápat jako odhad kvantilové funkce v bodě $p$, tedy $\hat{q}_p = \hat{Q}_n(p)$.

Z Centrální limitní věty víme, že pro empirickou distribuční funkci $\hat{F}_n(x)$ v pevném bodě $x$ platí:
\begin{equation}
    \sqrt{n}(\hat{F}_n(x) - F(x)) \xrightarrow{d} N(0, F(x)(1-F(x))).
\end{equation}

Abychom přešli od $\hat{F}_n$ ke $\hat{q}_p$, využijeme inverzní vztah. Aplikací funkcionální Delta metody (za předpokladu, že $F$ je diferencovatelná v $q_p$ a $f(q_p) > 0$) dostáváme pro kvantilový proces asymptotický vztah:
\begin{equation}
    \sqrt{n}(\hat{q}_p - q_p) \xrightarrow{d} N\left(0, \frac{\text{Var}(\mathbb{I}(X \le q_p))}{[f(q_p)]^2}\right).
\end{equation}

V čitateli zlomku je rozptyl indikátorové proměnné, která nabývá hodnoty 1 s pravděpodobností $p$ a 0 s pravděpodobností $1-p$. Její rozptyl je tedy $p(1-p)$.
Jmenovatel $[f(q_p)]^2$ plyne z derivace inverzní funkce (kvantilové funkce), neboť platí $(F^{-1})'(p) = \frac{1}{f(F^{-1}(p))} = \frac{1}{f(q_p)}$.

Výsledný asymptotický rozptyl výběrového kvantilu je tedy:
\begin{equation} \label{eq:avar}
    \text{AVar}(\hat{q}_p) = \frac{p(1-p)}{n [f(q_p)]^2}.
\end{equation}
Tato metoda je odvozena v \parencite{VanDerVaart1998}, ukazuje závislost přesnosti odhadu na pravděpodobnostní funkci. Je-li hustota $f(q_p)$ malá (např. v chvostech rozdělení), stává se jmenovatel velmi malým, což vede k velkým odhadům rozptylu.
Tento výsledek představuje standardní tvrzení asymptotické statistiky \parencite{VanDerVaart1998}. Ukazuje, že přesnost odhadu kvantilu závisí nepřímo úměrně hodnotě hustoty v daném bodě. V oblastech, kde je hustota nízká (chvosty rozdělení), je rozptyl odhadu kvantilu vysoký.
