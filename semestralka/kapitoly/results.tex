\chapter{Výsledky}
V této části prezentujeme výsledky Monte Carlo simulace. Sledujeme chování odhadů pro dvě hladiny kvantilů: $p=0.95$ a $p=0.99$.
Abychom zachovali konzistenci mezi odhadovaným kvantilem a intervalem spolehlivosti, konstruujeme pro kvantil $p=0.95$ interval spolehlivosti o hladině spolehlivosti 95 \% ($\alpha=0.05$) a pro extrémní kvantil $p=0.99$ interval o hladině 99 \% ($\alpha=0.01$).

Výsledné grafy jsou uspořádány do matice $2\times 3$:
\begin{itemize}
    \item \textbf{Řádky:} Horní řada odpovídá kvantilu $p=0.95$ (cílové pokrytí 0.95), dolní řada kvantilu $p=0.99$ (cílové pokrytí 0.99).
    \item \textbf{Sloupce:} Parametr asymetrie log-normálního rozdělení $\sigma \in \{0.5, 1.0, 1.5\}$.
\end{itemize}

\section{Pokrytí intervalů spolehlivosti (Coverage)}
Obrázek \ref{fig:coverage} zobrazuje pravděpodobnost pokrytí intervalů. Přerušovaná šedá čára značí nominální hladinu (0.95 pro horní řadu, 0.99 pro dolní řadu).

\begin{figure}[H]
    \centering
    \includegraphics[width=1.0\textwidth]{../simulace/coverage_probability.png}
    \caption{Coverage Probability. Horní řada: cíl 0.95 (95\% CI). Dolní řada: cíl 0.99 (99\% CI). Metody: Bootstrap (červená), Plug-in (modrá), Oracle (zelená).}
    \label{fig:coverage}
\end{figure}

Výsledky ukazují zřetelný rozdíl v chování metod v závislosti na parametrech simulace:
\begin{enumerate}
    \item \textbf{Oracle metoda (zelená)}: Tato referenční metoda konzistentně dosahuje nominálního pokrytí (0.95, resp. 0.99) ve všech scénářích. To potvrzuje, že samotná asymptotická aproximace (\ref{eq:avar}) je platná, pokud známe skutečnou hodnotu hustoty $f(q_p)$.
    
    \item \textbf{Plug-in metoda (modrá)}:
    \begin{itemize}
        \item V případě "běžných" dat ($\sigma \le 1.0$) poskytuje uspokojivé výsledky srovnatelné s Oracle metodou.
        \item V kritickém scénáři ($\sigma=1.5, p=0.99$) však dochází k dramatickému selhání. Pravděpodobnost pokrytí zde klesá hluboko pod cílovou hladinu 0.99 (často i pod 0.50). Graf relativního vychýlení (Obrázek \ref{fig:bias}) odhaluje příčinu: odhad směrodatné chyby je v tomto případě systematicky podhodnocen o více než 40 \%. Jádrový odhad hustoty v řídkém chvostu "přehlazuje" data, nadhodnocuje $f(q_p)$, a tím uměle snižuje odhadovaný rozptyl.
    \end{itemize}
    
    \item \textbf{Bootstrap (červená)}:
    \begin{itemize}
        \item Prokazuje mnohem vyšší robustnost vůči asymetrii. I v případě $\sigma=1.5$ si udržuje vyšší pravděpodobnost pokrytí než Plug-in metoda.
        \item U malých výběrů ($n=30$) však ani Bootstrap nedosahuje ideálního pokrytí 0.99. Zde narážíme na limity malého vzorku, kdy v kritické oblasti chvostu prostě "nejsou data" pro spolehlivé převzorkování.
    \end{itemize}
\end{enumerate}

\section{Systematické vychýlení (Relative Bias)}
Obrázek \ref{fig:bias} ukazuje relativní bias odhadu směrodatné chyby.

\begin{figure}[H]
    \centering
    \includegraphics[width=1.0\textwidth]{../simulace/relative_bias.png}
    \caption{Relativní vychýlení odhadu směrodatné chyby (SE).}
    \label{fig:bias}
\end{figure}

Z grafu je patrné, že se vzrůstajícím $\sigma$ se Plug-in metoda stává silně vychýlenou (záporný bias), což vysvětluje nízké pokrytí intervalů spolehlivosti. Bootstrap vykazuje menší vychýlení.

\section{Konvergence chyby (MSE)}
Vývoj střední čtvercové chyby (MSE) potvrzuje, že všechny metody jsou asymptoticky konzistentní (chyba klesá s $n$), ale rychlost konvergence se liší v závislosti na parametrech.

\begin{figure}[H]
    \centering
    \includegraphics[width=1.0\textwidth]{../simulace/mse_loglog.png}
    \caption{MSE v log-log měřítku.}
    \label{fig:mse}
\end{figure}
