\chapter{Metodika simulační studie}

\section{Generování dat}
Jako podkladová data používáme log-normální rozdělení $LN(\mu, \sigma^2)$ s parametry $\mu=0$ a $\sigma \in \{0.5, 1, 1.5\}$. Volba $\mu=0$ je bez újmy na obecnosti (jedná se o měřítkový parametr), zatímco různé hodnoty $\sigma$ nám umožňují studovat vliv šikmosti a těžkých chvostů rozdělení na přesnost odhadů. Zobrazení hustot pro uvažované parametry je uvedeno na Obrázku \ref{fig:densities}.
Generujeme náhodné výběry o rozsahu $n \in \{30, 100, 250, 500, 1000, 1500, 2000\}$. Tato škála pokrývá situace od velmi malých výběrů, kde asymptotické vlastnosti nemusí platit, až po velké výběry, kde očekáváme konvergenci k teoretickým hodnotám.

\begin{figure}[h]
    \centering
    \includegraphics[width=0.8\textwidth]{simulace/lognormal_densities.pdf}
    \caption{Hustoty Log-normálního rozdělení pro $\mu=0$ a různé hodnoty $\sigma$.}
    \label{fig:densities}
\end{figure}

Pro každý výběr odhadujeme tři kvantily reprezentující různé části rozdělení, medián (50\% kvantil), 95\% kvantil a 99\% kvantil. Počet replikací simulace byl stanoven na $M=1000$.

\section{Srovnávané metody}
V rámci studie porovnáváme tři přístupy k odhadu směrodatné chyby (SE) kvantilu:

\subsection{Asymptotická směrodatná chyba}
Tato metoda využívá znalosti skutečného rozdělení, ze kterého data pocházejí. Do vzorce (\ref{eq:avar}) dosazujeme skutečnou hustotu $f(q_p)$ log-normálního rozdělení.
$$ SE_{asymp} = \sqrt{\frac{p(1-p)}{n [f_{LN}(q_p)]^2}} $$
Tato hodnota představuje teoretickou asymptotickou směrodatnou chybu, ke které by se měly odhady s rostoucím rozsahem výběru blížit. Slouží nám jako referenční hodnota pro porovnání přesnosti ostatních metod.

\subsection{Praktická delta metoda}
Tato metoda je aplikovatelná v praxi, kdy neznáme skutečnou hustotu $f$. Místo ní použijeme její odhad $\hat{f}(q_p)$. V naší studii využíváme jádrový odhad hustoty (Kernel Density Estimation - KDE) s Gaussovským jádrem a Scottovým pravidlem pro volbu šířky vyhlazovacího okna (bandwidth) \parencite{Scott2015}.
$$ \widehat{SE}_{Plugin} = \sqrt{\frac{p(1-p)}{n [\hat{f}_{KDE}(\hat{q}_p)]^2}} $$
Jádrový odhad hustoty je definován jako:
\begin{equation}
    \hat{f}_{KDE}(x) = \frac{1}{nh} \sum_{i=1}^n K\left(\frac{x - X_i}{h}\right),
\end{equation}
kde $K$ je jádro (v našem případě Gaussovské jádro $K(u) = \frac{1}{\sqrt{2\pi}} e^{-u^2/2}$) a $h$ je šířka vyhlazovacího okna.
Nevýhodou je, že chyba odhadu hustoty se přenáší do chyby odhadu rozptylu kvantilu.

\subsection{Bootstrap}
Neparametrický bootstrap je metoda založená na převzorkování. Z původního výběru vytvoříme $R=5000$ bootstrapových výběrů (výběr s vracením), pro každý spočítáme výběrový kvantil $\hat{q}^*_p$ a rozptyl odhadujeme jako výběrový rozptyl těchto bootstrapových kvantilů.
$$ \widehat{SE}_{Boot} = \sqrt{\frac{1}{R-1} \sum_{r=1}^R (\hat{q}^*_{p,r} - \bar{q}^*_p)^2} $$
Tato metoda nevyžaduje explicitní odhad hustoty, ale je výpočetně náročnější \parencite{Horowitz2019}.

\subsection{Konstrukce intervalů spolehlivosti}
Pro všechny tři metody konstruujeme oboustranné intervaly spolehlivosti na hladině spolehlivosti $95\%$ ($\alpha = 0.05$). Využíváme asymptotické normality výběrového kvantilu (Waldův typ intervalu):
\begin{equation}
    CI_{0.95} = \left[ \hat{q}_p - z_{0.975} \cdot \widehat{SE}, \quad \hat{q}_p + z_{0.975} \cdot \widehat{SE} \right],
\end{equation}
kde $z_{0.975} \approx 1.96$ je $0.975$ kvantil standardizovaného normálního rozdělení. I pro metodu Bootstrap tedy v této studii využíváme normální aproximaci s odhadnutou směrodatnou chybou.

\section{Hodnotící kritéria}
Pro kvantitativní srovnání metod používáme následující kritéria, která hodnotí přesnost a spolehlivost odhadů. Označme $\theta$ skutečnou hodnotu parametru (např. rozptyl kvantilu) a $\hat{\theta}_m$ jeho odhad v $m$-té replikaci Monte Carlo simulace ($m=1,\dots,M$).

\subsection{Mean Squared Error (MSE)}
Střední čtvercová chyba měří celkovou přesnost odhadu, v níž je zahrnut jak rozptyl odhadu, tak jeho vychýlení. Je definována jako:
\begin{equation}
    \text{MSE}(\hat{\theta}) = \frac{1}{M} \sum_{m=1}^M (\hat{\theta}_m - \theta)^2
\end{equation}
Nízká hodnota MSE indikuje, že odhad je blízko skutečné hodnotě.

\subsection{Relative Bias (Relativní vychýlení)}
Relativní vychýlení vyjadřuje, o kolik procent metoda v průměru nadhodnocuje nebo podhodnocuje skutečnou hodnotu parametru.
\begin{equation}
    \text{RB}(\hat{\theta}) = \frac{\frac{1}{M} \sum_{m=1}^M \hat{\theta}_m - \theta}{\theta} \cdot 100\,\%
\end{equation}
Záporná hodnota RB značí systematické podhodnocení, což v kontextu odhadu rozptylu vede k příliš úzkým intervalům spolehlivosti. Kladná hodnota značí nadhodnocení.

\subsection{Pokrytí intervalu spolehlivosti (Coverage Probability - CP)}
Pokrytí intervalu spolehlivosti je pravděpodobnost, s jakou sestrojený interval spolehlivosti překryje skutečnou hodnotu odhadovaného parametru (kvantilu $q_p$).
\begin{equation}
    \text{CP} = \frac{1}{M} \sum_{m=1}^M \mathbb{I}(\hat{\theta}_{L,m} \le q_p \le \hat{\theta}_{U,m})
\end{equation}
kde $\mathbb{I}(\cdot)$ je indikátorová funkce a $[\hat{\theta}_{L,m}, \hat{\theta}_{U,m}]$ je interval spolehlivosti v $m$-té iteraci. Pro metodu s nominální hladinou spolehlivosti $1-\alpha$ (např. 0.95), by se CP měla blížit hodnotě $1-\alpha$. Výrazně nižší hodnota indikuje, že metoda není konzervativní (produkuje příliš mnoho chyb I. druhu).
