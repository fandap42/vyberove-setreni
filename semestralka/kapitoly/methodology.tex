\chapter{Metodika simulační studie}
Pro ověření přesnosti teoretického vzorce a srovnání s alternativními metodami jsme navrhli Monte Carlo simulační studii.

\section{Generování dat}
Jako podkladová data používáme log-normální rozdělení $LN(\mu, \sigma^2)$ s parametry $\mu=0$ a $\sigma=1$. Toto nastavení generuje data s pravostrannou asymetrií (šikmost $\approx 6.18$).
Generujeme náhodné výběry o rozsahu $n \in \{30, 100, 1000\}$ pro simulaci malých, středních a velkých datových souborů.

Pro každý výběr odhadujeme tři kvantily reprezentující různé části rozdělení:
\begin{itemize}
    \item $p=0.50$ (medián) - oblast s vysokou hustotou pravděpodobnosti.
    \item $p=0.95$ - začátek chvostu.
    \item $p=0.99$ - extrémní chvost, kde je hustota $f(q_p)$ velmi nízká.
\end{itemize}
Počet replikací simulation byl stanoven na $B=1000$.

\section{Srovnávané metody}
V rámci studie porovnáváme tři přístupy k odhadu směrodatné chyby (SE) kvantilu:

\subsection{1. Teoretický (Oracle) Taylor}
Tato metoda využívá znalosti skutečného rozdělení, ze kterého data pocházejí. Do vzorce (\ref{eq:avar}) dosazujeme skutečnou hustotu $f(q_p)$ log-normálního rozdělení.
$$ \widehat{SE}_{Oracle} = \sqrt{\frac{p(1-p)}{n [f_{LN}(q_p)]^2}} $$
Tato metoda slouží jako "zlatý standard" (benchmark), kterého v praxi nelze dosáhnout, ale ukazuje teoretickou mez přesnosti asymptotické aproximace.

\subsection{2. Praktický (Plug-in) Taylor}
Tato metoda je aplikovatelná v praxi, kdy neznáme skutečnou hustotu $f$. Místo ní použijeme její odhad $\hat{f}(q_p)$. V naší studii využíváme jádrový odhad hustoty (Kernel Density Estimation - KDE) s Gaussovským jádrem a Scottovým pravidlem pro volbu šířky vyhlazovacího okna (bandwidth).
$$ \widehat{SE}_{Plugin} = \sqrt{\frac{p(1-p)}{n [\hat{f}_{KDE}(\hat{q}_p)]^2}} $$
Nevýhodou je, že chyba odhadu hustoty se přenáší do chyby odhadu rozptylu kvantilu.

\subsection{3. Bootstrap}
Neparametrický bootstrap je metoda založená na převzorkování. Z původního výběru vytvoříme $R=200$ bootstrapových výběrů (výběr s vracením), pro každý spočítáme výběrový kvantil $\hat{q}^*_p$ a rozptyl odhadujeme jako výběrový rozptyl těchto bootstrapových kvantilů.
$$ \widehat{SE}_{Boot} = \sqrt{\frac{1}{R-1} \sum_{r=1}^R (\hat{q}^*_{p,r} - \bar{q}^*_p)^2} $$
Tato metoda nevyžaduje explicitní odhad hustoty, ale je výpočetně náročnější.

\section{Hodnotící kritéria}
Pro kvantitativní srovnání metod používáme následující kritéria, která hodnotí přesnost a spolehlivost odhadů. Označme $\theta$ skutečnou hodnotu parametru (např. rozptyl kvantilu) a $\hat{\theta}_m$ jeho odhad v $m$-té replikaci Monte Carlo simulace ($m=1,\dots,M$).

\subsection{Mean Squared Error (MSE)}
Střední čtvercová chyba měří celkovou přesnost odhadu, v níž je zahrnut jak rozptyl odhadu, tak jeho vychýlení. Je definována jako:
\begin{equation}
    \text{MSE}(\hat{\theta}) = \frac{1}{M} \sum_{m=1}^M (\hat{\theta}_m - \theta)^2
\end{equation}
MSE lze rozložit na složku rozptylu a kvadrát vychýlení: $\text{MSE} = \text{Var}(\hat{\theta}) + [\text{Bias}(\hat{\theta})]^2$. Nízká hodnota MSE indikuje, že odhad je blízko skutečné hodnotě.

\subsection{Relative Bias (Relativní vychýlení)}
Relativní vychýlení vyjadřuje, o kolik procent metoda v průměru nadhodnocuje nebo podhodnocuje skutečnou hodnotu parametru.
\begin{equation}
    \text{RB}(\hat{\theta}) = \frac{\frac{1}{M} \sum_{m=1}^M \hat{\theta}_m - \theta}{\theta} \cdot 100\,\%
\end{equation}
Záporná hodnota RB značí systematické podhodnocení (underestimation), což v kontextu odhadu rozptylu vede k příliš úzkým intervalům spolehlivosti. Kladná hodnota značí nadhodnocení (overestimation).

\subsection{Coverage Probability (Pravděpodobnost pokrytí)}
Coverage Probability (CP) je pravděpodobnost, s jakou sestrojený interval spolehlivosti (CI) překryje skutečnou hodnotu odhadovaného parametru (kvantilu $q_p$).
\begin{equation}
    \text{CP} = \frac{1}{M} \sum_{m=1}^M \mathbf{1}(\hat{\theta}_{L,m} \le q_p \le \hat{\theta}_{U,m})
\end{equation}
kde $\mathbf{1}(\cdot)$ je indikátorová funkce a $[\hat{\theta}_{L,m}, \hat{\theta}_{U,m}]$ je interval spolehlivosti v $m$-té iteraci. Pro metodu s nominální hladinou spolehlivosti $1-\alpha$ (např. 0.95), by se CP měla blížit hodnotě $1-\alpha$. Výrazně nižší hodnota indikuje, že metoda je antikonzervativní (produkuje příliš mnoho chyb I. druhu).
