\chapter{Závěr}
V této práci jsme odvodili asymptotický rozptyl výběrového kvantilu pomocí Taylorova rozvoje a porovnali jeho přesnost s metodou Bootstrap na datech z log-normálního rozdělení.

Simulační studie ukázala, že:
\begin{enumerate}
    \item Analytický vzorec (Taylor) funguje výborně pro centrální kvantily a dostatečně velké rozsahy výběrů ($n \ge 100$).
    \item Pro extrémní kvantily ($p=0.99$) a malé výběry ($n=30$) je použití analytického vzorce s odhadnutou hustotou (Plug-in) rizikové a často vede k nesprávným závěrům kvůli vysoké citlivosti na chybu odhadu hustoty ve chvostech.
    \item V případě malých výběrů a extrémních kvantilů nelze plně spoléhat ani na jednu z testovaných metod, ačkoliv Bootstrap vykazuje o něco lepší stabilitu.
\end{enumerate}

Pro praktické aplikace doporučujeme používat asymptotický vzorec obezřetně a v případě analýzy chvostů rozdělení ověřit výsledky pomocí robustnějších metod, jako je Bootstrap, nebo využít metody odvozené specificky pro teorii extrémních hodnot.
