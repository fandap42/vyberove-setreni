\documentclass[12pt,a4paper]{report} 
\usepackage{makra}
%TC:fileinclude \import dir,file

\usepackage[top=25mm,bottom=25mm,right=25mm,left=30mm,head=12.5mm,foot=12.5mm]{geometry}
\setlist[itemize]{noitemsep, topsep=0pt}
\setlist[enumerate]{noitemsep, topsep=0pt}
%\setlist[description]{noitemsep, topsep=0pt}



\linespread{1.2}
\parindent=0pt
\parskip=11pt

\def\NazevPrace{Využití Taylorova rozvoje ve výběrových šetřeních}
\def\TypPrace{SEMESTRÁLNÍ PRÁCE}
\def\Autor{František Pavlík}
\def\xname{pavf05}
\def\Semestr{ZS 2025/2026}
\def\Predmet{4ST414 - Teorie výběrových šetření}
\def\Program{Statistika}


\begin{document}

\import{template/}{titulka}

\begin{abstract}
    \noindent Práce se zabývá aplikací Taylorova rozvoje v teorii výběrových šetření, konkrétně jako nástroje pro odhad rozptylu nelineárních statistik. V kontextu komplexních výběrových plánů, které zahrnují stratifikaci a shlukový výběr, je standardní výpočet rozptylu pro statistiky jako poměry či regresní koeficienty netriviální. Práce představuje teoretický základ a odvození metody Taylorovy linearizace (známé též jako metoda delta). Tento přístup aproximuje nelineární funkci odhadů pomocí její lineární aproximace prvního řádu. Tím je problém odhadu rozptylu složité statistiky transformován na výpočetně zvládnutelný problém odhadu rozptylu lineárního odhadu (úhrnu), pro který existují zavedené postupy. Práce se věnuje detailnímu odvození metody pro poměrové a regresní odhady a její implementace ve statistických sofwarech.
    
\end{abstract}

\newpage

\printbibliography[title={Použitá literatura},heading={bibintoc}]

\end{document}